\section{Conformal blocks}
    The start of our discussion of conformal blocks is the very classical fact that, given a variety $X$ (over $\bbC$), the adic completion of each stalk $\scrO_{X, x}$ at a \textit{regular closed} point $x \in X$ is isomorphic to $\bbC(\!(v_1, ..., v_{\dim X})\!)$, which is to be thought of as a choice of coordinates for the algebra of \say{holomorphic} functions on the formal punctured polydisc centered around $x$. For our purposes, we will be interested only in the case:
        $$\dim X = 1$$
    i.e. $X$ is a curve, as only dimension $1$ do we have that the field of fractions $\scrK_{X, x}$ of the $\m_{X, x}$-adic completions $\hat{\scrO}_{X, x}$ (where $\m_{X, x}$ is the unique maximal ideal of $\scrO_{X, x}$) are the $\m_{X, x}$-adic completions of the field of globally rational functions on $X$, which shall be denoted by $\scrK_X$. Also, for the sake of simplicity, let us suppose that $X$ is smooth (so that all points are regular), irreducible (hence connected), and projective (so that $X$ can be specified by $\scrK_X$ alone).
    
    Next, at each such closed point $x \in X$, let us have a lattice $L_x$ of rank $\dim X$ and let us define:
        $$\frakH_x := (L_x \tensor_{\Z} \bbC)^* \tensor_{\bbC} \scrK_{X, x}$$
    Then, let us define the completed Heisenberg algebra associated to $L_x$ to be the central extension:
        $$0 \to \bbC c_x \to \scrH_x \to \frakH_x \to 0$$
    corresponding to the residue form as in definition \ref{def: lattice_heisenberg_algebras}. Let us say that a $\scrH_x$-module $V$ is \textbf{smooth} if $\m_{X, x}$ acts nilpotently on $V$. As such, given an identification $L \cong L_x$ at some closed point $x \in X$, one shall then get an equivalence:
        $$\left\{ \text{$\tilde{\scrH}_L$-modules on which $\tilde{\scrH}_L^+$ acts nilpotently} \right\}$$
        $$\cong$$
        $$\left\{ \text{smooth $\scrH_x$-modules} \right\}$$
    
    Now, notice that for any lattice $L$ and any weight $\lambda \in \h_L^* := (L \tensor_{\Z} \bbC)^{**} \cong L \tensor_{\Z} \bbC$, the vacuum representation $\V_L^{\lambda}$ of highest-weight $\lambda$ of the Weyl algebra $\tilde{\scrH}_L$ has, by construction (see definition \ref{def: lattice_weyl_algebras}), the property that, the upper triangular topological subalgebra $\tilde{\scrH}_L^+ \subset \tilde{\scrH}_L$ acts topologically nilpotently on $\V_L^{\lambda}$. Therefore, given an isomorphism of lattices $L \cong L_x$, the $\tilde{\scrH}_L$-module $\V_L^{\lambda}$ will also have a smooth $\scrH_x$-module structure, and it is thus possible to associate to each closed point $x \in X$ the VOA:
        $$\V_x := \bigoplus_{\lambda \in L_x} \V_{L_x}^{\lambda}$$

    \subsection{Coinvariants and conformal blocks}
        Let $\punc{\scrH}_x := \Gamma(X \setminus x, \scrO_{X \setminus x})$ be the algebra of regular functions on $X$ minus a closed point $x$. This embeds as a commutative subalgebra into $\scrK_{X, x}$ by mapping functions $f \in \punc{\scrH}_x$ to their Laurent series expansion around $x$, and it can be shown that there is an induced embedding of (abelian) Lie algebras $\punc{\scrH}_x \subset \scrH_x$. Each of the $\tilde{\scrH}_x := \tilde{\scrH}_{L_x}$-modules $\V_x^{\lambda} := \V_{L_x}^{\lambda}$ thus gains a $\punc{\scrH}_x$-module structure.

        Now, let us observe that for any lattice $L$, because the $\rmU(\tilde{\h}_L)$-ideal generate by $c_L - 1$ is actually a coideal, the quotient $\bar{\scrH}_L := \rmU(\tilde{\h}_L)/(c_L - 1)$ is actually a Hopf algebra quotient of $\rmU(\tilde{\h}_L)$, whose Hopf structure is given by:
            $$\Box(H) := H \tensor 1 + 1 \tensor H \pmod{(c_L - 1)}$$
            $$S(H) := -H$$
        for all $H \in \tilde{\h}_L$. The Weyl algebra $\tilde{\scrH}_L$, which is the adic completion of $\bar{\scrH}_L$ by definition, is thus a topological Hopf algebra given in the same way.

        \begin{definition}[Coinvariants] \label{def: coinvariants}
            Let $U$ be a coalgebra and $V$ be a $U$-comodule, say determined by a coaction $\rho: V \to U \tensor_{\bbC} V$. An element $v \in V$ is said to be \textbf{$U$-coinvariant} if and only if $\rho(v) = u \tensor v$ for some $u \in U$. The sub-comodule of $V$ consisting of $U$-coinvariant elements is denoted by ${}^UV$.
        \end{definition}
        Of course, if $U$ is a bialgebra then by duality, $({}^UV)^*$ (or the graded-dual, in case $V$ is graded and $U$ coacts homogeneously) will be isomorphic to $(V^*)^U$, the $U$-submodule of $V^*$ consisting of $U$-invariant elements, i.e. those $v \in V$ such that $u \cdot v = v$ for all $u \in U$. 
        \begin{example}[Group-like elements]
            Let $U$ be a coalgebra with comultiplication $\Box: U \to U \tensor_{\bbC} U$. Group-like elements of $U$ are those $u \in U$ such that $\Box(u) = u \tensor u$, so they are $U$-coinvariant should we view $U$ as a comodule over itself.
        \end{example}

        For our purposes, it is more convenient to define coinvariants dually, in particular, by firstly defining so-called \say{conformal blocks}.
        \begin{definition}[Conformal blocks] \label{def: conformal_blocks}
            The $\punc{\scrH}_x$-module of \textbf{conformal blocks} at $x \in X$, denoted by $\Lambda_x^{\lambda}$, shall be given by:
                $$\Lambda_x^{\lambda} := ( ( \V_x^{\lambda} )^* )^{\punc{\scrH}_x}$$
            (note that we are taking the \textit{full} linear dual). 
        \end{definition}
        \begin{proposition}[Conformal blocks and coinvariants] \label{prop: conformal_blocks_and_coinvariants}
            The $\punc{\scrH}_x$-comodule $V_x^{\lambda} := (\Lambda_x^{\lambda})^*$ embeds into $\V_x^{\lambda}$ as the sub-comodule of $\punc{\scrH}_x$-coinvariants. Furthermore, this embedding is a section of the canonical quotient map $\V_x^{\lambda} \to \V_x^{\lambda}/\punc{\scrH}_x \cdot \V_x^{\lambda}$.
        \end{proposition}
            \begin{proof}
                
            \end{proof}

    \subsection{Correlation functions}